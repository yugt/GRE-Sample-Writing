\section{Issue 16}
\paragraph{
Governments must ensure that their major cities receive the financial support they need in order to thrive, because it is primarily in cities that a nation's cultural traditions are preserved and generated.
}
\subsection{Score 6}


The speaker's claim is actually threefold: (1) ensuring the survival of large cities and, in turn, that of cultural traditions, is a proper function of government; (2) government support is needed for our large dries and cultural traditions to survive and thrive; and (3) cultural traditions are preserved and generated primarily in our large cities.
I strongly disagree with all three claims.


First of all, subsidizing cultural traditions is not a proper role of govemment.
Admittedly, certain objectives, such as public health and safety, are so essential to the survival of large dries and of nations that government has a duty to ensure that they are met.
However, these objectives should not extend tenuously to preserving cultural traditions.
Moreover, government cannot possibly play an evenhanded role as cultural patron.
Inadequate resources call for restrictions, priorities, and choices.
It is unconscionable to relegate normative decisions as to which cities or cultural traditions are more deserving, valuable, or needy to a few legislators, whose notions about culture might be misguided or unrepresentative of those of the general populace.
Also, legislators are all too likely to make choices in favor of the cultural agendas of their home towns and states, or of lobbyists with the most money and influence.


Secondly, subsidizing cultural traditions is not a necessary role of government.
A lack of private funding might justify an exception.
However, culture--by which I chiefly mean the fine arts--has always depended primarily on the patronage of private individuals and businesses, and not on the government.
The Medicis, a powerful banking family of Renaissance Italy, supported artists Michelangelo and Raphael.
During the 20th Century the primary source of cultural support were private foundations established by industrial magnates Carnegie, Mellon,.


Rockefeller and Getty.
And tomorrow cultural support will come from our new technology and media moguls----including the likes of Ted Turner and Bill Gates.
In short, philanthropy is alive and well today, and so government need not intervene to ensure that our cultural traditions are preserved and promoted.


Finally, and perhaps most importantly, the speaker unfairly suggests that large cities serve as the primary breeding ground and sanctuaries for a nation's cultural traditions.
Today a nation's distinct cultural traditions--its folk art, crafts, traditional songs, customs and ceremonies--burgeon instead in small towns and rural regions.
Admittedly, our cities do serve as our centers for "high art"; big cities are where we deposit, display, and boast the world's preeminent art, architecture, and music.
But big-city culture has little to do any- more with one nation's distinct cultural traditions.
After all, modern cities are essentially multicultural stew pots; accordingly, by assisting large cities a government is actually helping to create a global culture as well to subsidize the traditions of other nations' cultures.


In the final analysis, government cannot philosophically justify assisting large cities for the purpose of either promoting or preserving the nation's cultural traditions; nor is government assistance necessary toward these ends.
Moreover, assisting large cities would have little bearing on our distinct cultural traditions, which abide elsewhere.
