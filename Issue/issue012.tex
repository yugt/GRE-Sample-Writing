\section{Issue 12}
\paragraph{
No field of study can advance significantly unless outsiders bring their knowledge and experience to that field of study.
}
\subsection{Score 6}


I strongly agree with the assertion that significant advances in knowledge require expertise from various fields.
The world around us presents a seamless web of physical and anthropogenic forces, which interact in ways that can be understood only in the context of a variety of disciplines.
Two examples that aptly illustrate this point involve the fields of cultural anthropology and astronomy.


Consider how a cultural anthropologist's knowledge about an ancient civilization is enhanced not only by the expertise of the archeologist--who unearths the evidence--but ultimately by the expertise of biochemists, geologists, linguists, and even astronomers.
By analyzing the hair, nails, blood and bones of mummified bodies, biochemists and forensic scientists can determine the life expectancy, general well-being, and common causes of death of the population.
These experts can also ensure the proper preservation of evidence found at the archeological site.
A geologist can help identify the source and age of the materials used for tools, weapons, and structures--thereby enabling the anthropologist to extrapolate about the civilization's economy, trades and work habits, life styles, extent of travel and mobility, and so forth.
Linguists are needed to interpret hieroglyphics and extrapolate from found fragments of writings.
And an astronomer can help explain the layout of an ancient city as well as the design, structure and position of monuments, tombs, and temples--since ancients often looked to the stars for guidance in building cities and structures.


An even more striking example of how expertise in diverse fields is needed to advance knowledge involves the area of astronomy and space exploration.
Significant advancements in our knowledge of the solar system and the universe require increasingly keen tools for observation and measurement.
Telescope technology and the measurement of celestial distances, masses, volumes, and so forth, are the domain of astrophysicists.


These advances also require increasingly sophisticated means of exploration.
Manned and unmanned exploratory probes are designed by mechanical, electrical, and computer engineers.
And to build and enable these technologies requires the acumen and savvy of business leaders, managers, and politicians.
Even diplomats might play a role--insofar as major space projects require intemafional cooperative efforts among the world's scientists and governments.
And ultimately it is our philosophers whose expertise helps provide meaning to what we learn about our universe.


In sum, no area ofinteUectual inquiry operates in a vacuum.
Because the sciences are inextricably related, to advance our knowledge in any one area we must understand the interplay among them all.
Moreover, it is our non-scienfists who make possible the science, and who bring meaning to what we learn from it.
