\section{Issue 23}
\paragraph{
Most people would agree that buildings represent a valuable record of any society's past, but controversy arises when old buildings stand on ground that modern planners feel could be better used for modern purposes.
In such situations, modern development should be given precedence over the preservation of historic buildings so that contemporary needs can be served.
}
\subsection{Score 6}


The speaker asserts that wherever a practical, utilitarian need for new buildings arises this need should take precedence over our conflicting interest in preserving historic buildings as a record of our past.
In my view, however, which interest should take precedence should be determined on a case-by-case basis--and should account not only for practical and historic considerations but also aesthetic ones.


In determining whether to raze an older building, planners should of course consider the community's current and anticipated utilitarian needs.
For example, if an additional hospital is needed to adequately serve the health-care needs of a fast-growing community, this compelling interest might very well outweigh any interest in preserving a historic building that sits on the proposed site.
Or if additional parking is needed to ensure the economic survival of a city's downtown district, this interest might take precedence over the historic value of an old structure that stands in the way of a parking structure.
On the other hand, if the need is mainly for more office space, in some cases an architecturally appropriate add-on or annex to an older building might serve just as well as razing the old building to make way for a new one.
Of course, an expensive retrofit might not be worthwhile if no amount of retrofitting would meet the need.


Competing with a community's utilitarian needs is an interest preserving the historical record.
Again, the weight of this interest should be determined on a case-by-case basis.
Perhaps an older building uniquely represents a bygone era, or once played a central role in the city's history as a municipal structure.
Or perhaps the building once served as the home of a founding family or other significant historical figure, or as the location of an important historical event.
Any of these scenarios might justify saving the building at the expense of the practical needs of the community.
On the other hand, if several older buildings represent the same historical era just as effectively, or if the building's history is an unremarkable one, then the historic value of the building might pale in comparison to the value of a new structure that meets a compelling practical need.


Also competing with a community's utilitarian needs is the aesthetic and architectural value of the building itself--apart from historical events with which it might be associated.
A building might be one of only a few that represents a certain architectural style.
Or it might be especially beautiful, perhaps as a result of the craftsmanship and materials employed in its construction--which might be cost-prohibitive to replicate today.
Even retrofitting the building to accommodate current needs might undermine its aesthetic as well as historic value, by altering its appearance and architectural integrity.
Of course it is difficult to quantify aesthetic value and weigh it against utilitarian considerations.
Yet planners should strive to account for aesthetic value nonetheless.


In sum, whether to raze an older building in order to construct a new one should never be determined indiscriminately.
Instead, planners should make such decisions on a case-by-case basis, weighing the community's practical needs against the building's historic and aesthetic value.
