\section{Issue 3}

\paragraph{Some people believe that corporations have a responsibility to promote the well-being of the socities and environments in which they operate.
Others believe that the only responsibility of corporations, provided they operate within the law, is to make as much money as possible.}

\subsection{Score 6}
It is not uncommon for some to argue that, in the world in which we live, corporations have a responsibility to society and to the environment in which they operate.
\emph{Proponents} of this view would argue that major environmental catastrophes (e.g., the oil spill in the Gulf) are key examples of the damage that can be wrought when corporations are allowed to operate unchecked.
Yet within that very statement lies a contradiction that undermines this kind of thinking--it is necessary for outside forces to check the behavior of corporations, because we do not expect corporations to behave in such manner.
In fact, the expectation is simply that corporations will follow the law, and in the course of doing so, engage in every possible tactic to their advantage in the pursuit of more and greater profit.
To expect otherwise from corporations is to fail to understand their purpose and their very structure.

The corporation arose as a model of business in which capital could be raised through the contributions of stockholders: investors purchase shares in a company, and their money is then used as the operating capital for the company.
Shareholders buy stock not because they are hoping to better make the world a better place or because they have a desire to improve the quality of life but because they expect to see a return in their investment in this company.
The company may itself have generally \emph{altruistic} goals (perhaps it is a think tank that advises the government on how to improve relations with the Middle East, or perhaps it is a company built around finding alternative forms of energy), but the immediate expectation of the investor is that himself will see dividends, or profits, from the investment he has made.
This is even more true in the case of companies that are purely profit driven and which do not have goals that are particularly directed toward social environment--a description that applies to the vast majority of corporations.

Is it a bad thing to have a corporation negatively affect the environment (and by extension, its inhabitants)?
To pump noxious fumes into the atmosphere as a by-product of its manufacturing process?
Of course, and this is why agencies such as the EPA were established and why governments--federal, state and local--are expected to monitor such companies.
Any and all corporations should be expected to temper their input pursuit of profit with the necessity of following those safeguards that have been legislated as protections.
But the assumption that corporations have a inherent obligation or responsibility to go above and beyond that to actively PROMOTE the environment and the well-being of society is absurd.

Engaging in practices to adhere to legal expectations to protect society and the environment is costly to corporations.
If the very purpose of a corporation is to generate profits, and the obligation to adhere to safety expectations established by law cuts into those profits, then to expect corporations to embrace such practices beyond what is required to presume that they willingly engage in an inherently self-destructive process: the unnecessary lowering of profits.
This is \emph{antithetical} to the very concept of the corporation.
Treehuggers everywhere should be pleased that environmental protections exist, but to expect corporations to ``make the world a better place'' is to embrace altruism to the point that it becomes \emph{delusion}.

This is not to say that we should reject efforts to hold corporations accountable.
In fact, the opposite is true--we should be \emph{vigilant} with the business world and maintain our expectations that corporations do not make their profits at the EXPENSE of the well-being of society.
But that role must be fulfilled by a watchdog, not the corporation itself, and those expectations mush be imposed UPON the corporations, not expected FROM them.

\subsection{Score 5}
In order to survive, corporations must make money.
Successful corporations try and make as much money as possible.
Yet this incentive to make money does not mean that a corporation can be \emph{detriment} to the society in which it operates.
Corporations have a duty and a responsibility to ensure the well being of the society in which they are a part.

Contributing to the well being of a society is actually beneficial to a corporation in many cases.
One of these is making sure that workers are well taken care of.
Absenteeism and neglect while on duty are a big problem for corporations, as is attracting the best workers, who hopefully will lower the risks caused by absenteeism and neglect.

One way that corporations can attract these workers is by offering them generous benefits.
If, for example, an employer includes with enployment a good health plan, they will be able to attract better workers than one that does not, and that will aid the corporation greatly.
Health care plans provided by employers mean that these people have at their disposal health coverage, which means that they have the care they need if they get sick.
This also might encourage perventive care, something that has been shown to reduce the cost and risk of developing other major ailments.

Another area where corporations providing support for themselves and society is in the creation of human capital.
Globolization and increased education means that employers need a better educated workforce more than ever.
One way that employers can contribute to this is by sponsoring worker training programs, or paying for their emplyees to return to school.
This creates a more educated workforce for employers, as well as may increase the loyalty of employees to an employer.
An employee who received an education sponsored by an employer may be thankful for receiving that education, and may work harder for that employer.
This creates a benefit for employers and employees.

The main reason that corporations have a duty to contribute to the well being of society is that they are a part of the society.
Even though they have an economic desire to make a profit, corporations also should think long term about actions they take which could hurt their company.
A good example of this is BP, after the recent oil spill in the gulf.
Their desire to make a profit meant that they did not keep up on all of their safety regulations and standard, and the result of the then faulty of equipment caused a massive spill.
This cost them huge amounts of money to clean up, as well as the fines they had to pay for causing this.
The biggest loss for BP however is that their brand name will be associated in the US and abroad as the company that caused this guabt oil spill.
As that spill was happening, many people boycotted that company, resulting in lost potential revenue.
They may realize that as they lose business to people upset by the spill, that making sure a spill didn't happen in the first place was cheaper.

Another reason corporations have to ensure the well-being of a society is that by making a society better off, a company may have more consumers.
This is especially true for corporations that sell goods for middle and upper class consumers.
If a corporation tries to bring people up and increase the overall economic well being of society, they may find that more and more people have the ability to afford their goods.
This could generate huge new profitsd for this corporation, since their pool of potential consumers has gone up considerably.
Concentrating on the long term here means that corporations can increase their pool of potential consumers.

By denying responsibility to a society, a corporation is only looking as the possible short term profits, not the potential long term ones.
While in the short term it may work for a corporation to ignore their societal responsibility, it is adventageous in the long term for the entire corporation to make sure society is getting better.
The potential for new markets, products, production processes and other beneficial factors that come from promoting well being is quite large.
This is something that corporations should be ready and willing to take advantage of, and something that society should hold them accountable for.