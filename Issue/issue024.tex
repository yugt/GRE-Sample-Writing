\section{Issue 24}
\paragraph{
Students should memorize facts only after they have studied the ideas, trends, and concepts that help explain those facts.
Students who have learned only facts have learned very little.
}
\subsection{Score 6}


The speaker makes a threshold claim that students who learn only facts learn very little, then condudes that students should always learn about concepts, ideas, and trends before they memorize facts.
While I wholeheartedly agree with the threshold claim, the condusion unfairly generalizes about the learning process.
In fact, following the speaker's advice would actually impede the learning of concepts and ideas, as well as impeding the development of insightful and useful new ones.


Turning first to the speaker's threshold daim, I strongly agree that ifwe learn only facts we learn very little.
Consider the task of memorizing the periodic table of dements, which any student can memorize without any knowledge of chemistry, or that the table relates to chemistry.
Rote memorization of the table amounts to a bit of mental exercise-an opportunity to practice memorization techniques and perhaps learn some new ones.
Otherwise, the student has learned very little about chemical dements, or about anything for that matter.


As for the speaker's ultimate claim, I concede that postponing the memorization of facts until after one leams ideas and concepts holds certain advantages.
With a conceptual framework already in place a student is better able to understand the meaning of a fact, and to appreciate its significance.
As a result, the student is more likely to memorize the fact to begin with, and less likely to forget it as time passes.
Moreover, in my observation students whose first goal is to memorize facts tend to stop there--for whatever reason.
It seems that by focusing on facts first students risk equating the learning process with the assimilation of trivia; in turn, students risk learning nothing of much use in solving real world problems.


Conceding that students must learn ideas and concepts, as well as facts relating to them, in order to learning anything meaningful, I nevertheless disagree that the former should always precede the latter--for three reasons.
In the first place, I see know reason why memorizing a fact cannot precede learning about its meaning and significance--as long as the student does not stop at rote memorization.
Consider once again our hypothetical chemistry student.
The speaker might advise this student to first learn about the historical trends leading to the discovery of the elements, or to learn about the concepts of altering chemical compounds to achieve certain reactions--before studying the periodic table.
Having no familiarity with the basic vocabulary of chemistry, which includes the informarion in the periodic table, this student would come away from the first two lessons bewildered and confused in other words, having learned little.


In the second place, the speaker misunderstands the process by which we learn ideas and concepts, and by which we develop new ones.
Consider, for example, how economics students learn about the relationship between supply and demand, and the resulting concept of market equilibrium, and of surplus and shortage.
Learning about the dynamics of supply and demand involves (1) entertaining a theory, and perhaps even formulating a new one, (2) testing hypothetical scenarios against the theory, and (3) examining real-world facts for the purpose of confirming, refuting, modifying, or qualifying the theory.
But which step should come first? The speaker would have us follow steps 1 through 3 in that order.
Yet, theories, concepts, and ideas rarely materialize out of thin air; they generally emerge from empirical observations--i.e., facts.
Thus the speaker's notion about how we should learn concepts and ideas gets the learning process backwards.


In the third place, strict adherence to the speaker's advice would surely lead to illconceived ideas, concepts, and theories.
Why? An idea or concept conjured up without the benefit of data amounts to little more than the conjurer's hopes and desires.
Accordingly, conjurers will tend to seek out facts that support their prejudices and opinions, and overlook or avoid facts that refute them.
One telling example involves theories about the center of the universe.


Understandably, we ego-driven humans would prefer that the universe revolve around us.
Early theories presumed so for this reason, and facts that ran contrary to this ego-driven theory were ignored, while observers of these facts were scorned and even vilified.
In short, students who strictly follow the speaker's prescription are unlikely to contribute significantly to the advancement of knowledge.


To sum up, in a vacuum facts are meaningless, and only by filling that vacuum with ideas and concepts can students learn, by gaining useful perspectives and insights about facts.
Yet, since facts are the very stuff from which ideas, concepts, and trends spring, without some facts students cannot learn much of anything.
In the final analysis, then, students should learn facts right along with concepts, ideas, and trends.
